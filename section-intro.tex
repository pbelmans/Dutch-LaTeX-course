\begin{frame}
  \frametitle{Aanpak van de cursus}

  Vier lessen, de woensdagmiddagen 16, 23 en 30 november en 7 december telkens van 2 tot 5. Er is een pauze voorzien.

  Ik voorzie slides waarin in kort alles toelicht, voor meer informatie verwijs ik naar The Not So Short Introduction to \LaTeXe, meer bepaald de Nederlandse versie die te downloaden is op~\url{http://ctan.org/tex-archive/info/lshort/dutch}. Vanaf nu:~\texttt{lshort-dutch}. Andere goede bronnen:
  \begin{enumerate}
    \item Wikibooks: \url{http://en.wikibooks.org/wiki/LaTeX}
    \item TeX.SX: \url{http://tex.stackexchange.com}
    \item Google
  \end{enumerate}
\end{frame}

\begin{frame}
  \frametitle{Opbouw van de cursus}

  Min of meer gelijklopend met \texttt{lshort-dutch} maar met mijn eigen voorkeuren, de planning:

  \begin{description}[tweede week]
    \item[eerste week] inleiding, installatie, eerste stappen
    \item[tweede week] bibliografie\"en, uitzicht, tabellen en figuren
    \item[derde week] wiskunde, ``gevorderde'' technieken
    \item[vierde week] naar keuze (waarschijnlijk het (mooi) invoegen van code en het maken van presentaties)
  \end{description}
\end{frame}

\begin{frame}
  \frametitle{Geruststelling}

  Ik ga \emph{geen} geschiedenisles geven. Voor de ge\"interesseerden, ik heb een TechTalk (of \TeX talk zo u wil) gegeven hierover: zie \url{http://acmantwerp.acm.org/wp-content/uploads/2010/10/textalk.pdf}
\end{frame}

\begin{frame}
  \frametitle{Bedoeling van de cursus}

  Op een paar lessen alles vertellen dat ik na meerdere jaren \LaTeX\ gebruiken heb geleerd is onmogelijk, daarom:
  \begin{itemize}
    \item op gang helpen: zelf (basis)documenten kunnen maken;
    \item \emph{zelfredzaam} worden: er is enorm veel informatie te vinden op het internet;
    \item \emph{best practices}: dingen waarvan ik nu denk dat ik ze zelf vroeger had moeten leren maar die te weinig aan bod komen in de meeste tutorials.
  \end{itemize}
\end{frame}

\begin{frame}
  \frametitle{Waarom \LaTeX?}

  Tijd voor een eerste stukje interactie. Shoot.
\end{frame}


\begin{frame}
  \frametitle{Deze slides}
  Ik hou me aan volgende conventie:\pause
  \begin{alertblock}{Waarschuwing}
	  Every time you ignore this, God kills a kitten.
  \end{alertblock}
  
  \pause\begin{exampleblock}{Opmerking}
	  Hierin vertel ik wat er onderhuids gebeurt, geef ik alternatieven, \ldots
  \end{exampleblock}
\end{frame}
