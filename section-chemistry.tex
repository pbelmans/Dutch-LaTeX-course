\begin{frame}
  \frametitle{Disclaimer}

  Ik ben geen chemicus, bijgevolg is mijn ervaring eerder beperkt (om niet te zeggen onbestaande). Als ik onzin uitkraam, gelieve me daar op te wijzen.

  Deze slides zijn dus ook maar een korte introductie.
\end{frame}

\subsection{Chemische formules}
\begin{frame}[fragile]
  \frametitle{\package{mhchem}}

  \begin{minted}{tex}
\ce{H2O}, \ce{[AgCl2]-}, \ce{1/2H2O}, \ce{^{227}_{90}Th+}
  \end{minted}
  resulteert in \ce{H2O}, \ce{[AgCl2]-}, \ce{1/2H2O}, \ce{^{227}_{90}Th+}
  \begin{minted}{tex}
\ce{CO2 + C -> 2CO}, \ce{CO2 + C ->[\alpha][\beta] 2CO}
  \end{minted}
  resulteert in \ce{CO2 + C -> 2CO}, \ce{CO2 + C ->[\alpha][\beta] 2CO}
\end{frame}

\begin{frame}[fragile]
  \frametitle{Groter voorbeeld}

  \begin{minted}{tex}
\ce{Zn^2+
  <=>[\ce{+ 2OH-}][\ce{+ 2H+}]
  $\underset{\text{zinkaation}}{\ce{Zn(OH)2 v}}$
  <=>C[+2OH-][{+ 2H+}]
  $\underset{\text{tetrahydroxozinkaation}}{\cf{[Zn(OH)4]^2-}}$
}
  \end{minted}
\ce{Zn^2+
  <=>[\ce{+ 2OH-}][\ce{+ 2H+}]
  $\underset{\text{zinkaation}}{\ce{Zn(OH)2 v}}$
  <=>C[+2OH-][{+ 2H+}]
  $\underset{\text{tetrahydroxozinkaation}}{\cf{[Zn(OH)4]^2-}}$
}
\end{frame}

\begin{frame}[fragile]
  \frametitle{Werking}

  Er is een \verb|\ce{}|-commando dat op een heel natuurlijke wijze chemische reacties interpreteert. Je hoeft geen math mode te gebruiken maar voor bijvoorbeeld sub- en superscripts is gebeurt dit wel standaard (dus gebruik \verb|\text{}| indien nodig).

  Om chemische formules binnen wiskundeomgevingen te gebruiken: \verb|\cee{}|.

  \begin{minted}{tex}
\usepackage[version=3]{mhchem}
  \end{minted}
\end{frame}


\subsection{Chemische structuren}
\begin{frame}
  \frametitle{Chemische structuren}

  Een paar mogelijkheden: \package{xymtex}, \package{ppchtex} of \package{tikz}. En er is \package{chemstyle} voor de consistentie.

  Een andere mogelijkheid: tekenen in pakweg ChemDraw en exporteren naar \texttt{pdf}.
\end{frame}


\subsection{SI-eenheden}
\begin{frame}
  \frametitle{\package{siunitx}}

  De meest recente package voor consistent werken met eenheden. Ik weet niet of deze werkt op de pc's hier, maar we bekijken de documentatie eens.
\end{frame}
