\begin{frame}[fragile]
  \frametitle{Geschiedenis}

  Een van \emph{de} redenen om \TeX\ te ontwikkelen: tot de jaren '80 was het onmogelijk om zelf goed wiskunde te typesetten.

  De American Mathematical Society heeft een hoop werk geleverd in de jaren '90, met als resultaat:
  \begin{minted}{tex}
\usepackage{amsmath}
  \end{minted}

  \begin{exampleblock}{Andere omgevingen}
    Ik ga enkel \package{amsmath} omgevingen gebruiken, er zijn ook \TeX\ en \LaTeX-varianten maar deze zijn niet per se nodig.
  \end{exampleblock}
\end{frame}

\begin{frame}[fragile]
  \frametitle{Math mode}

  Normaal gezien houdt \LaTeX\ zich bezig met het typesetten van doorlopende tekst (dus in \emph{text mode}), omdat wiskunde zo'n fundamenteel andere aanpak vereist moetten we \LaTeX\ vertellen dat we naar \emph{math mode} willen.
  
  \inputminted{tex}{math-example.tex}
\end{frame}

\begin{frame}
  \frametitle{Het resultaat}

  In doorlopende tekst: $a^2+b^2=c^2$.

\begin{equation}
  f^{(n)}(a)=\frac{n!}{2\pi i}\oint_{\Gamma}
  \frac{f(\zeta)}{(\zeta-z)^{n+1}}\mathrm{d}\zeta
\end{equation}


  De eerste is \emph{inline} (of in \emph{text style}), de andere is in \emph{display style}.
\end{frame}

\begin{frame}[fragile]
  \frametitle{Spatiegebruik en de rol van letters}

  \begin{enumerate}
    \item spaties en regeleindes worden \emph{volledig} genegeerd (waar mogelijk), witruimte in formules komt door de semantiek van symbolen of handmatig;
    \item gewone letters zijn variabelen en worden dus in italic geplaatst.
      \begin{LTXexample}
$dit klopt niet$ \\
$f_{\text{int}}$
      \end{LTXexample}
  \end{enumerate}
\end{frame}

\subsection{Binnen math mode}
\begin{frame}[fragile]
  \frametitle{Grieks alfabet}

  Heel simpel:\\
  \texttt{\textbackslash$\langle$\textsl{letter}$\rangle$} en \texttt{\textbackslash$\langle$\textsl{Letter}$\rangle$}

  \begin{LTXexample}
$\alpha\beta\Gamma\Delta$
  \end{LTXexample}

  \begin{alertblock}{A tiny catch}
    Sommige Griekse hoofdletters zijn hetzelfde als de Latijnse, deze hebben dus geen aparte letter. Er is dus geen \verb|\Alpha| of \verb|\Chi|.
  \end{alertblock}
\end{frame}

\begin{frame}[fragile]
  \frametitle{Varianten}

  Voor de liefhebber, een paar varianten (de drie eerste zijn frequent, de rest niet):

  \begin{LTXexample}
$\epsilon\,\varepsilon$ \\
$\theta\,\vartheta$ \\
$\phi\,\varphi$ \\
$\pi,\varpi$ \\
$\rho\,\varrho$ \\
$\sigma\,\varsigma$
  \end{LTXexample}
\end{frame}

\begin{frame}[fragile]
  \frametitle{Operators}

  Ook wel ``functies met een naam''. Zoals daar zijn $\cos$, $\sin$, $\log$, \ldots

  \begin{LTXexample}
$\sin(\pi)=0$,
$\tan(\pi/2)=1$.
  \end{LTXexample}

  De meest voor de hand liggende zijn voor gedefinieerd, anders: \\
  \texttt{\textcolor{uagreen}{\textbackslash DeclareMathOperator}\textbackslash\textsl{$\langle$commandonaam$\rangle$}\{\textsl{$\langle$naam$\rangle$}\}}
  \begin{minted}{tex}
\DeclareMathOperator\characteristic{char}
\DeclareMathOperator\trace{tr}
\end{minted}
\end{frame}

\begin{frame}[fragile]
  \frametitle{Subscript mogelijk maken}

  \begin{LTXexample}
\begin{equation}
  \lim_{x\rightarrow 0}
  \frac{\sin x}{x}
\end{equation}
  \end{LTXexample}

  \texttt{\textcolor{uagreen}{\textbackslash DeclareMathOperator}\textbackslash\textsl{$\langle$commandonaam$\rangle$}\{\textsl{$\langle$naam$\rangle$}\}}
  \begin{minted}{tex}
\DeclareMathOperator*{\argmax}{arg\,max} 
\DeclareMathOperator*\argmax{arg\,max} 
  \end{minted}
\end{frame}

\begin{frame}[fragile]
  \frametitle{Sub- en superscript}

  Daarom zijn \verb|_| en \verb|^| dus speciale karakters:
  \small
  \begin{LTXexample}
$F_{n+2}=F_{n+1}+F_n$ \\
$O(n^{n^n})$ \\
$f|_{[0,1]}$ \\
$\lim_{x\rightarrow 0}$
\begin{equation}
  \int_0^x\!f'(x)\,\mathrm{d}x
\end{equation}
  \end{LTXexample}
\end{frame}

\begin{frame}[fragile]
  \frametitle{Breuken en binomiaalco\"effici\"enten}

  \texttt{\textcolor{uagreen}{\textbackslash frac}\{$\langle$\textsl{p}$\rangle$\}\{$\langle$\textsl{q}$\rangle$\}} en \texttt{\textcolor{uagreen}{\textbackslash binom}\{$\langle$\textsl{p}$\rangle$\}\{$\langle$\textsl{q}$\rangle$\}}

  \begin{LTXexample}
$\frac{\sin(x)}{\cos(x)}$ \\
$\binom{n}{k}$ \\
$\displaystyle\binom{n}{k}$
\begin{equation}
  \frac{\sin(x)}{\cos(x)}
\end{equation}
  \end{LTXexample}
  Of met~\package{xfrac}: $\sfrac{1}{2}$.
\end{frame}

\begin{frame}[fragile]
  \frametitle{\texttt{\textbackslash displaystyle}}

  Er zijn 4 formaten, van groot naar klein: \texttt{\textcolor{uagreen}{\textbackslash displaystyle}}, \texttt{\textcolor{uagreen}{\textbackslash textstyle}}, \texttt{\textcolor{uagreen}{\textbackslash scriptstyle}}, \texttt{\textcolor{uagreen}{\textbackslash scriptscriptstyle}}. Respectievelijk in display mode (in equations etc.), text mode (inline, of binnen omgevingen in equations), eerste niveau van sub- en superscripts, alle volgende niveaus van sub- en superscript.
\end{frame}

\begin{frame}[fragile]
  \frametitle{Wortels}

  \begin{LTXexample}
$\sqrt{2}$,~$\sqrt[3]{n}$
\begin{equation}
  \sqrt{\frac{3}{5}}
\end{equation}
  \end{LTXexample}
\end{frame}

\begin{frame}[fragile]
  \frametitle{Sommen en aanverwanten}

  \small
\begin{LTXexample}
$\sum_{k=1}^n k^2$ \\
\begin{equation}
  \sum_{k=1}^n k^2
  \bigotimes \coprod
  \oint \iiint
\end{equation}
\end{LTXexample}

\begin{alertblock}{\texttt{\textbackslash big}-varianten}
  Leer op de goede momenten \texttt{\textcolor{uagreen}{\textbackslash} bigcup} gebruiken. Want $\bigcup_{i=1}^n U_i\neq \cup_{i=1}^n U_i$.
\end{alertblock}
\end{frame}

\begin{frame}[fragile]
  \frametitle{\texttt{\textbackslash substack\{\}}}

  \begin{minted}{tex}
\begin{equation}
  \sum_{\substack{0<k<m \\ 0<l<n}} k+l
\end{equation}
  \end{minted}
\begin{equation}
  \sum_{\substack{0<k<m \\ 0<l<n}} k+l
\end{equation}
\end{frame}

\begin{frame}[fragile]
  \frametitle{\texttt{\textbackslash mathclap\{\}}}

  \begin{minted}{tex}
\begin{equation}
  \sum_{\mathclap{\substack{0<k<m \\ 0<l<n}}}\,k+l
\end{equation}
  \end{minted}
\begin{equation}
  \sum_{\mathclap{\substack{0<k<m \\ 0<l<n}}}\,k+l
\end{equation}
\end{frame}

\begin{frame}[fragile]
  \frametitle{Haakjes}

  \begin{LTXexample}
$(\cdot) \, [\cdot] \,$
$\{\cdot\}\, |\cdot| \,$
$\|\cdot\| \,$
$\langle\cdot\rangle \,$
$\lfloor\cdot\rfloor \,$
$\lceil\cdot\rceil$
  \end{LTXexample}
\end{frame}

\begin{frame}[fragile]
  \frametitle{Automatisch aanpassen}

  \begin{minted}{tex}
\begin{gather}
  \left( \frac{1}{2} \right) \\
  \left\{ x\in\mathbb{Q} \middle| x<\frac{1}{2} \right\}
\end{gather}
\end{minted}
\begin{gather}
  \left( \frac{1}{2} \right) \\
  \left\{ x\in\mathbb{Q} \,\middle|\, x<\frac{1}{2} \right\}
\end{gather}
\end{frame}

\begin{frame}[fragile]
  \frametitle{Matrices}

  \begin{minted}{tex}
\begin{gather}
  \begin{matrix} 1 & 0 \\ 0 & 1 \end{matrix} \quad
  \begin{pmatrix} 1 & 0 \\ 0 & 1 \end{pmatrix}
\end{gather}
  \end{minted}
\begin{gather}
  \begin{matrix} 1 & 0 \\ 0 & 1 \end{matrix} \quad
  \begin{pmatrix} 1 & 0 \\ 0 & 1 \end{pmatrix}
\end{gather}

Maar er zijn ook nog \verb|bmatrix|, \verb|Bmatrix|, \verb|vmatrix| en \verb|Vmatrix|. Ook is er telkens een \texttt{\textsl{$\langle$d$\rangle$}matrix*}-variant om alignment te bepalen.
\end{frame}

\begin{frame}[fragile]
  \frametitle{Grote matrices}
  
  \footnotesize
  \begin{minted}{tex}
\begin{gather}
  A_{m,n} =
    \begin{pmatrix}
      a_{1,1} & a_{1,2} & \cdots & a_{1,n} \\
      a_{2,1} & a_{2,2} & \cdots & a_{2,n} \\
      \vdots  & \vdots  & \ddots & \vdots  \\
      a_{m,1} & a_{m,2} & \cdots & a_{m,n}
    \end{pmatrix} 
\end{gather}
  \end{minted}
\begin{gather}
  A_{m,n} =
    \begin{pmatrix}
      a_{1,1} & a_{1,2} & \cdots & a_{1,n} \\
      a_{2,1} & a_{2,2} & \cdots & a_{2,n} \\
      \vdots  & \vdots  & \ddots & \vdots  \\
      a_{m,1} & a_{m,2} & \cdots & a_{m,n}
    \end{pmatrix} 
\end{gather}
\end{frame}

\begin{frame}[fragile]
  \frametitle{Tekst}

  Wat niet te doen:
  \begin{LTXexample}
$\dim_{Goldie}$
  \end{LTXexample}
  maar wel:
  \begin{LTXexample}
$\dim_{\text{Goldie}}$
  \end{LTXexample}

  Ook formatted text (zie eerste les).
\end{frame}

\begin{frame}
  \frametitle{Wiskundige lettertypes}

  \texttt{\textcolor{uagreen}{\textbackslash mathnormal}\{\}}, \texttt{\textcolor{uagreen}{\textbackslash mathrm}\{\}}, \texttt{\textcolor{uagreen}{\textbackslash mathit}\{\}} en \texttt{\textcolor{uagreen}{\textbackslash mathbf}\{\}} doen wat ze suggereren.

  \begin{description}
    \item[\texttt{\textcolor{uagreen}{\textbackslash mathsf}\{\}}] sans-serif, $\text{$k$-}\mathsf{Vect}$ (maar nu zie je weinig)
    \item[\texttt{\textcolor{uagreen}{\textbackslash mathcal}\{\}}] calligraphic, $\mathcal{O}$
    \item[\texttt{\textcolor{uagreen}{\textbackslash mathfrak}\{\}}] Fraktur, $\mathfrak{m}$
    \item[\texttt{\textcolor{uagreen}{\textbackslash mathbb}\{\}}] blackboard bold, $\mathbb{R}$
  \end{description}

  De laatste twee zitten verstopt in \package{amsfonts}.
\end{frame}

\begin{frame}[fragile]
  \frametitle{Accenten}

  Nu geen constructies meer met backslashes en obscure symbolen, maar dingen van de vorm \texttt{\textcolor{uagreen}{\textbackslash hat}\{\textsl{$\langle$x$\rangle$}\}}, \texttt{\textcolor{uagreen}{\textbackslash widehat}\{\textsl{$\langle$xyz$\rangle$}\}} en \texttt{\textcolor{uagreen}{\textbackslash overline}\{\textsl{$\langle$xyz$\rangle$}\}} wat resulteert in $\hat{x}$, $\widehat{xyz}$ en $\overline{xyz}$.

  De gewone weglatingstekens gelden nog wel: \verb|x'| levert $x'$.
\end{frame}

\begin{frame}
  \frametitle{Horizontale ruimte}

  Als basis: \texttt{\textcolor{uagreen}{\textbackslash quad}} is gelijk aan de fontgrootte, dus 10 tot 12 punt. \texttt{\textcolor{uagreen}{\textbackslash qquad}} is dit twee keer.

  Kleiner:
  \begin{description}
    \item[\texttt{\textbackslash,}] 3/18 quad
    \item[\texttt{\textbackslash:}] 4/18 quad
    \item[\texttt{\textbackslash;}] 5/18 quad
    \item[\texttt{\textbackslash!}] -3/18 quad
  \end{description}

  \begin{equation}
    \oint_\Gamma \zeta(x)\,\mathrm{d}x
  \end{equation}
\end{frame}

\begin{frame}
  \frametitle{Puntjes}

  In alle soorten en maten: \texttt{\textcolor{uagreen}{\textbackslash dots}}, \texttt{\textcolor{uagreen}{\textbackslash ldots}}, \texttt{\textcolor{uagreen}{\textbackslash cdots}}, \texttt{\textcolor{uagreen}{\textbackslash vdots}}, \texttt{\textcolor{uagreen}{\textbackslash ddots}}.

  {\ }

  Er zijn ook semantische varianten, \texttt{\textcolor{uagreen}{\textbackslash dotsc}} (komma's), \texttt{\textcolor{uagreen}{\textbackslash dotsb}} (binaire operaties), \texttt{\textcolor{uagreen}{\textbackslash dotsm}} (multiplicatie), \texttt{\textcolor{uagreen}{\textbackslash dotsi}} (integraal), \texttt{\textcolor{uagreen}{\textbackslash dotso}} (de rest).
\end{frame}

\begin{frame}[fragile]
  \frametitle{Symbolen}

  \emph{Bijna} elk gekend symbool in de wiskunde heeft een bepaald commando:
  \small
  \begin{LTXexample}
$\leq$, $\subseteq$, $\perp$
  \end{LTXexample}

  \normalsize
  Zij die niet in standaard \TeX\ zitten zijn wel in een package te vinden dan: The Comprehensive \LaTeX\ Symbol List. Vind je het niet zo gauw? Hulde voor \url{http://detexify.kirelabs.org}.
\end{frame}

\begin{frame}
  \frametitle{Oefening}

  Zoek volgende symbolen op en typeset ze:
  \begin{itemize}
	\item $\Gamma$
	\item $\subsetneqq$
	\item $\triangleleft$
	\item $\varinjlim$ (injectieve limiet)
	\item $\boxplus$
  \end{itemize}
\end{frame}

\begin{frame}[fragile]
  \frametitle{Pijlen}

  \texttt{\textcolor{uagreen}{\textbackslash}$\langle$\textsl{lengte}$\rangle\langle$\textsl{richting}$\rangle$\textcolor{uagreen}{arrow}}

  \begin{description}
    \item[\textsl{richting}] \texttt{right}, \texttt{left}, \texttt{up}, \texttt{down}
    \item[\textsl{lengte}] voeg \texttt{long} toe indien nodig
  \end{description}

  Wil je een dubbele pijl? Zet de eerste letter in hoofdletters.
  \begin{LTXexample}
$\rightarrow\ \uparrow\ $
$\leftarrow\ \downarrow$ \\
$\Rightarrow$
$\longleftarrow$
$\Leftrightarrow$
  \end{LTXexample}
\end{frame}


\subsection{Omgevingen}
\begin{frame}[fragile]
  \frametitle{Allemaal goed en wel\ldots}

  maar hoe geef ik nu wiskunde in? Je zag al wat hints: inline gebruiken we \verb|$...$| en er is zoiets als
  \begin{minted}{tex}
\begin{equation}
  ...
  \label{equation:...}
\end{equation}
  \end{minted}
\end{frame}

\begin{frame}[fragile]
  \frametitle{Een overzicht}

  \begin{description}
    \item[\texttt{equation}] een losstaande vergelijking
    \item[\texttt{align}] een reeks uitdrukkingen die uitgelijnd worden
    \item[\texttt{gather}] een reeks uitdrukkingen
  \end{description}

  De varianten met een ster laten de nummering vallen.
\end{frame}

\begin{frame}[fragile]
  \frametitle{\texttt{align} voorbeeld}

  \footnotesize
  \begin{minted}{tex}
\begin{align*}
  (x-y)^3&=(x-y)(x-y)^2 \\
  &=(x-y)(x^2-2xy+y^2) \\
  &=x^3-3x^2y+3xy^2-3y^3
\end{align*}
  \end{minted}
\begin{align*}
  (x-y)^3&=(x-y)(x-y)^2 \\
  &=(x-y)(x^2-2xy+y^2) \\
  &=x^3-3x^2y+3xy^2-3y^3
\end{align*}

Je kan ook meerdere alignments characters gebruiken.
\end{frame}

\begin{frame}[fragile]
  \frametitle{Verwijzen}

  Het gekende truukje met \texttt{\textcolor{uagreen}{\textbackslash label}\{\}} maar nu met \texttt{\textcolor{uagreen}{\textbackslash eqref}}.

  En als je wil herstarten met nummeren:
  \begin{minted}{tex}
\numberwithin{equation}{section}
  \end{minted}
\end{frame}

\begin{frame}[fragile]
  \frametitle{\texttt{cases}}

  Stuksgewijze definities:
  \begin{minted}{tex}
\begin{equation}
 u(x) =
  \begin{cases}
    \exp{x} & \text{voor $x\geq 0$} \\
    1 & \text{voor $x<0$}
  \end{cases}
\end{equation}
  \end{minted}
\begin{equation}
 u(x) =
  \begin{cases}
    \exp{x} & \text{if $x\geq 0$} \\
    1 & \text{if $x < 0$}
  \end{cases}
\end{equation}
\end{frame}

\begin{frame}[fragile]
  \frametitle{Stellingen}

  Voor al uw stellingen: \'e\'en package, \package{amsthm}.

  \texttt{\textcolor{uagreen}{\textbackslash newtheorem}\{$\langle$\textsl{omgeving}$\rangle$\}[$\langle$\textsl{nummering}$\rangle$]\{$\langle$\textsl{naam}$\rangle$\}} en \texttt{\textcolor{uagreen}{\textbackslash newtheorem}\{$\langle$\textsl{omgeving}$\rangle$\}\{$\langle$\textsl{naam}$\rangle$\}}[$\langle$\textsl{reset}$\rangle$]

  \begin{minted}{tex}
\newtheorem{theorem}{Stelling}[section]
\newtheorem{corollary}[theorem]{Gevolg}
\newtheorem{definition}[theorem]{Definitie}
\newtheorem{example}[theorem]{Voorbeeld}
\newtheorem{lemma}[theorem]{Lemma}
\newtheorem{remark}[theorem]{Opmerking}
  \end{minted}
\end{frame}

\begin{frame}[fragile]
  \frametitle{Opmerkingen bij \package{amsthm}}

  \begin{itemize}
    \item drie stijlen, \texttt{\textcolor{uagreen}{\textbackslash theoremstyle}\{definition\}};
    \item er is ook een omgeving voor bewijzen:
      {\footnotesize
      \begin{minted}{tex}
\begin{theorem}[Stelling van Pythagoras]
  $a^2+b^2=c^2$.

  \begin{proof}
    Triviaal.  
  \end{proof}
\end{theorem}
      \end{minted}
      }
    \item \texttt{\textcolor{uagreen}{\textbackslash numberswapping}}
  \end{itemize}
\end{frame}
  
