\documentclass[a4paper]{article}
\usepackage[fleqn,leqno]{amsmath}
\usepackage[dutch]{babel}
\usepackage{colonequals}
\usepackage[version=3]{mhchem}

\setlength{\parindent}{0em}

\title{Opgave derde les}
\author{$\langle$jouw naam$\rangle$}

\begin{document}
\maketitle


\begin{equation}
  \label{eq:p_suc_r_W}
	P_{\text{success}}^{(r)}(W_1, W_2) \colonequals \sum_{i=1}^{k_1} (-1)^{i+1} \dbinom{k_1}{i} \left(\frac{\dbinom{\nu-i}{k_1}}{\dbinom{\nu}{k_1}}\right)^{W_1-1}\left(\frac{\dbinom{\nu-i}{k_2}}{\dbinom{\nu}{k_2}}\right)^{W_2}
\end{equation}

\begin{equation}
	\label{eq:p_suc_bibd_W_weighted}
	P_{\text{success}}^{(i)}(W_1, W_2) \colonequals \sum_{i=1}^{k_1} (-1)^{i+1} \dbinom{k_1}{i} \frac{\dbinom{(b_1-1)-iS}{W_1-1}}{\dbinom{b_1-1}{W_1-1}}	\sum_{N_i}\frac{|N_i|}{\displaystyle\sum_{N_i}|N_i|}\frac{\dbinom{b_2-N_i}{W_2}}{\dbinom{b_2}{W_2}}
\end{equation}

En nu een paar pijlen:~$\longmapsto$,~$\Updownarrow$,~$\searrow$, \ldots.

Wat chemie
\begin{equation}
  \ce{Hg^2+ ->[\ce{I-}]
  $\underset{\mathrm{red}}{\ce{HgI2}}$
  ->C[I-]
  $\underset{\mathrm{red}}{\ce{[Hg^{II}I4]^2-}}$
  }
\end{equation}

\end{document}
