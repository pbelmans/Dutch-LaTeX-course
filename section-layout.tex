\begin{frame}[fragile]
  \frametitle{Fontgrootte}

  Eerst en vooral: globaal via de opties van je documentclass.
\begin{minted}{tex}
\documentclass[a4paper,11pt]{article}
\end{minted}
\end{frame}

\begin{frame}[fragile]
  \frametitle{Lokaal wijzigingen aanbrengen}

  Er zijn 10 commando's die je tekst relatief tegenover de globale grootte wijzigen, gebruik deze dus het best.
\end{frame}

\begin{frame}[fragile]
\begin{tabular}{llll}
  \toprule
  size & 10pt (default) & 11pt option & 12pt option \\\midrule
  \verb|\tiny| &	6.80565 &	7.33325 &	7.33325 \\
  \verb|\scriptsize|& 	7.97224 &	8.50012& 	8.50012  \\
  \verb|\footnotesize|& 	8.50012 &	9.24994 &	10.00002 \\
  \verb|\small| 	&9.24994 &	10.00002 &	10.95003         \\
  \verb|\normalsize|& 	10.00002 &	10.95003 &	11.74988 \\
  \verb|\large| 	&11.74988 &	11.74988 &	14.09984         \\
  \verb|\Large| 	&14.09984 &	14.09984 &	15.84985         \\
  \verb|\LARGE| 	&15.84985 &	15.84985 &	19.02350         \\
  \verb|\huge| 	&19.02350 &	19.02350 &	22.82086         \\
  \verb|\Huge| 	&22.82086 &	22.82086 &	22.82086        \\
  \bottomrule
  \end{tabular}
\end{frame}

\begin{frame}
  \frametitle{Lettertypes}

  We stoten op een van de nadelen van \LaTeX, in de jaren '80 was nog niet alles zo flexibel als we nu gewend zijn. Spelen met verschillende lettertypes is in \LaTeX~niet zo gemakkelijk als we zouden willen.
\end{frame}

\begin{frame}
  \frametitle{Elk nadeel heb z'n voordeel}

  \begin{exampleblock}{Tip}
	Gebruik steeds zo weinig mogelijk verschillende (en exotische) lettertypes.
  \end{exampleblock}

  Mijn advies: verander je lettertype enkel globaal.
\end{frame}

\begin{frame}[fragile]
  \frametitle{En hoe dan te veranderen?}

  Het is niet mogelijk om de lettertypes uit je Windows- of Linuxinstallatie te gebruiken, enkel lettertypes waarvan \LaTeX~de informatie die het nodig heeft bezit kunnen werken.

  Daarnaast is het normaal gezien alleen maar mogelijk om \emph{globaal} te wijzigen. Een voorbeeld:
\begin{minted}{tex}
\usepackage[T1]{fontenc}
\usepackage{tgpagella}
\end{minted}
\end{frame}

\begin{frame}[fragile]
  \frametitle{En wat als ik toch lokale wijzigingen wil?}

  Gebruik KOMA-Script of \texttt{memoir}, dit zijn aparte document classes (zie vorige les) die veel meer flexibiliteit bieden.
\end{frame}

\begin{frame}
  \frametitle{En waar vind ik dan die lettertypes?}

  E\'en plek: \url{http://www.tug.dk/FontCatalogue/}
\end{frame}

\begin{frame}[fragile]
  \frametitle{\XeTeX}

  Dit is een uitbreiding op \LaTeX~die w\'el goed met lettertypes overweg kan. Maar ik ben er te weinig mee vertrouwd om er iets zinnig over te kunnen zeggen.
\end{frame}
