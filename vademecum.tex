\documentclass[a4paper,9pt,article,oneside,openany]{memoir}
\usepackage[dutch]{babel}
\usepackage{minted}

% fonts
\usepackage[T1]{fontenc}
\usepackage[charter]{mathdesign}
\usepackage[scaled]{beramono,berasans}
\usepackage{microtype}

% colors
\definecolor{uablue}{RGB}{0,61,100}
\definecolor{uared}{RGB}{126,0,47}
\definecolor{vividbrown}{RGB}{215,154,70}
\definecolor{uagreen}{RGB}{0,126,17}

\author{Pieter Belmans}
\title{\LaTeX-vademecum}

\begin{document}
\maketitle

\chapter{Eerste les}
\section{Basisdocument}
\begin{minted}{tex}
\documentclass[a4paper]{article}
\usepackage[dutch]{babel}
\usepackage{hyperref}

\author{Naam}
\title{Titel}

\begin{document}
\maketitle

\end{document}
\end{minted}
Alles tussen \mint{tex}|\documentclass[a4paper]{article}| en \mint{tex}|\begin{document}| is de \emph{preamble}, alles tussen \mint{tex}|\begin{document}| en \mint{tex}|\end{document}| het eigenlijke document. In de preamble komen de packages en de configuratie van je bestand, in het document je tekst.

\section{Witruimte}
\begin{enumerate}
  \item \'e\'en spatie is hetzelfde als meerdere spaties;
  \item spaties, tabs en enters zijn (bijna) hetzelfde;
  \item spaties aan het begin van een regel worden volledig genegeerd;
  \item twee regeleindes, dus \'e\'en witregel is een nieuwe alinea.
\end{enumerate}

\section{Speciale karakters}
Sommige karakters in \LaTeX\ zijn niet zomaar te produceren, je moet ze \emph{escapen}.

\begin{center}
  \begin{tabular}{cccccccccccccccc}
    \mint{tex}|\$| & \$ & 
    \mint{tex}|\&| & \& &
    \mint{tex}|\%| & \% &
    \mint{tex}|\#| & \# &
    \mint{tex}|\_| & \_ &
    \mint{tex}|\{| & \{ &
    \mint{tex}|\}| & \}
  \end{tabular}
\end{center}

Ook voor aanhalingstekens is het opletten geblazen: gebruik \verb|``tekst''| in plaats van \verb|"tekst"|. Dit geldt ook voor \verb|`enkele'|. Daarnaast zijn er vier liggende streepjes:
\begin{center}
  \begin{tabular}[]{cccc}
  	naam & input & output & doel \\\midrule
    hyphen & \mint{tex}|-| & - & splitsen in lettergrepen \\
  	en dash & \mint{tex}|--| & -- & reeksen jaartallen, pagina's, \ldots \\
  	em dash & \mint{tex}|---| & --- & gedachtestreepje \\
  	minteken & \mint{tex}|$-$| & $-$ & in wiskundige uitdrukking
  \end{tabular}
\end{center}

\section{Structuur}
Je maakt een nieuwe alinea door een witregel open te laten, \emph{niet} door creatief gebruik van \mint{tex}|\\|. Om hoofdstukken en varianten aan te duiden gebruiken we
\begin{center}
  \begin{tabular}{lcl}
    commando & niveau & opmerking \\\midrule
    \mint{tex}|\part{}| & -1 & niet in~\texttt{letter} \\
    \mint{tex}|\chapter{}| & 0 & niet in~\texttt{letter} en~\texttt{article} \\
    \mint{tex}|\section{}| & 1 & niet in~\texttt{letter} \\
    \mint{tex}|\subsection{}| & 2 & niet in~\texttt{letter} \\
    \mint{tex}|\subsubsection{}| & 3 & niet in~\texttt{letter} \\
    \mint{tex}|\paragraph{}| & 4 & niet in~\texttt{letter} \\
    \mint{tex}|\subparagraph{}| & 5 & niet in~\texttt{letter} \\
  \end{tabular}
\end{center}

En gebruik \mint{tex}|\tableofcontents| om een inhoudstafel te produceren. Als je niet wil dat een bepaald hoofdstuk hier in komt: \mint{tex}|\section*{}|, de zogeheten \emph{starred variant}.

\end{document}
