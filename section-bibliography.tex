\subsection{Bib\TeX}
\begin{frame}
  \frametitle{Bibliografiebeheer}

  Een van d\'e redenen om~\LaTeX\ te gebruiken is het goede beheer van referenties en bibliografie\"en. We gebruiken hiervoor~Bib\TeX. Dit is een extern programma dat output genereert die we kunnen gebruiken in~\LaTeX.

  \begin{exampleblock}[Bib\LaTeX]
    Er is een opvolger: Bib\LaTeX, maar omdat ik nog niet zeker weet of en hoe deze gaat werken in MiK\TeX~2.9 laat ik deze dit jaar nog achterwege.
  \end{exampleblock}
\end{frame}

\begin{frame}
  \frametitle{Concept}

  \begin{itemize}
    \item \'e\'en (of meerdere) bestanden waarin bibliografie\-informatie staat, onafhankelijk van de weergave;
    \item citeren gebeurt zoals we met~\texttt{\textcolor{uagreen}{\textbackslash label}\{$\langle$\textsl{key}$\rangle$\}} en \texttt{\textcolor{uagreen}{\textbackslash ref}\{$\langle$\textsl{key}$\rangle$\}} werkten, maar nu met~\texttt{\textcolor{uagreen}{\textbackslash cite}\{$\langle$\textsl{referentie}$\rangle$\}};
    \item de layout van de bibliografie en de citaten is aanpasbaar en onafhankelijk van hoe we onze bibliografie hebben opgeslagen.
  \end{itemize}
\end{frame}

\begin{frame}
  \frametitle{\texttt{.bib} bestand}

  \begin{enumerate}
    \item bevat (delen van) de bibliografie;
    \item volgt een vaste structuur:

      \texttt{@\textsl{$\langle$entry-type$\rangle$}\{\textsl{$\langle$key$\rangle$},\\
      \ \ \textsl{$\langle$field-1$\rangle$} = \textsl{$\langle$value-1$\rangle$}, \\
      \ \ \ldots \\
      \ \ \textsl{$\langle$field-$n\rangle$} = \textsl{$\langle$value-$n\rangle$}, \\
      \}}
      
      en we zullen deze nu invullen.
  \end{enumerate}
\end{frame}

\begin{frame}
  \frametitle{Entry types}

  Bepaalt wat voor type het document heeft:
  \begin{itemize}
    \item \texttt{article}
    \item \texttt{book}
    \item \texttt{inbook}: een specifiek deel van een boek
    \item \texttt{misc}
    \item \texttt{proceedings}
    \item \ldots
  \end{itemize}
\end{frame}

\begin{frame}
  \frametitle{Voorbeeld}

\texttt{@book\{\textsl{$\langle$key$\rangle$},\\
\ \ \textsl{$\langle$field-1$\rangle$} = \textsl{$\langle$value-1$\rangle$}, \\
\ \ \ldots \\
\ \ \textsl{$\langle$field-$n\rangle$} = \textsl{$\langle$value-$n\rangle$}, \\
\}}
\end{frame}

\begin{frame}
  \frametitle{Keys}

  Zoals met~\texttt{\textcolor{uagreen}{\textbackslash label}\{\textsl{$\langle$key$\rangle$}\}} hebben we een unieke sleutel nodig. Je kan deze opbouwen op basis van de auteur, het jaar van publicatie, de titel, \ldots\ Zorg dat het voor jou duidelijk is.
\end{frame}

\begin{frame}
  \frametitle{Voorbeeld}

\texttt{{@}book\{formalized-music,\\
\ \ \textsl{$\langle$field-1$\rangle$} = \textsl{$\langle$value-1$\rangle$}, \\
\ \ \ldots \\
\ \ \textsl{$\langle$field-$n\rangle$} = \textsl{$\langle$value-$n\rangle$}, \\
\}}
\end{frame}

\begin{frame}[fragile]
  \frametitle{Velden en waarden}

  Elke entry bevat een aantal keys (of velden) met geassocieerde waarden, \emph{afhankelijk van het type}. Sommige zijn verplicht, anderen zijn optioneel. Enkele voorbeelden:
  \begin{itemize}
    \item \texttt{author}
    \item \texttt{title}
    \item \texttt{year}
    \item \ldots
  \end{itemize}
  \verb|author = {Iannis Xenakis},| \\
  \verb|title = {Formalized Music},|
\end{frame}

\begin{frame}
  \frametitle{Types en velden}

  \begin{description}
    \item[\texttt{article}] verplicht: \texttt{author}, \texttt{title}, \texttt{journal}, \texttt{year} \\
      optioneel: \texttt{volume}, \texttt{number}, \texttt{pages}, \texttt{month}, \texttt{note}, \texttt{key}
    \item[\texttt{book}] verplicht: \texttt{author} of~\texttt{editor}, \texttt{title}, \texttt{publisher}, \texttt{year} \\
      optioneel: \texttt{volume}, \texttt{series}, \texttt{address}, \texttt{edition}, \texttt{month}, \texttt{note}, \texttt{key}
    \item[\texttt{inbook}] verplicht: \texttt{author} of~\texttt{editor}, \texttt{title}, \texttt{chapter} of~\texttt{pages}, \texttt{publisher}, \texttt{year} \\
      optioneel: \texttt{volume}, \texttt{series}, \texttt{address}, \texttt{edition}, \texttt{month}, \texttt{note}, \texttt{key}
    \end{description}
\end{frame}

\begin{frame}
  \frametitle{Types en velden (2)}

  \begin{description}[\texttt{proceedings}]
    \item[\texttt{misc}] verplicht: geen \\
      optioneel: \texttt{author}, \texttt{title}, \texttt{howpublished}, \texttt{month}, \texttt{year}, \texttt{note}, \texttt{key}
    \item[\texttt{proceedings}] verplicht: \texttt{title}, \texttt{year} \\
    optioneel: \texttt{editor}, \texttt{publisher}, \texttt{organization}, \texttt{address}, \texttt{month}, \texttt{note}, \texttt{key}
  \end{description}
\end{frame}

\begin{frame}
  \frametitle{Een overzicht graag}

  \small
  \url{http://newton.ex.ac.uk/tex/pack/bibtex/btxdoc/node6.html}, \url{http://newton.ex.ac.uk/tex/pack/bibtex/btxdoc/node7.html}
  {\ } \\
  {\ } \\
  \url{http://en.wikibooks.org/wiki/LaTeX/Bibliography_Management}
\end{frame}

\begin{frame}[fragile]
  \frametitle{Bib\TeX\ voorbeeld}

  \begin{minted}{tex}
@book{formalized-music,
  author = {Iannis Xenakis},
  title = {Formalized Music}
  publisher = {Pendragon Press},
  year = {2001},
  edition = {2},
  isbn = {978-1576470794},
}
  \end{minted}
\end{frame}

\begin{frame}[fragile]
  \frametitle{En websites?}

  In Bib\TeX\ geen concrete oplossing, wel een pseudo-oplossing:
  \begin{minted}{text}
@misc{wiki:aardappel,
  author = {Wikipedia}
  title = {Aardappel},
  year = {2011},
  howpublished =
    {\url{http://nl.wikipedia.org/wiki/Aardappel}}
}
  \end{minted}
  Vergeet geen~\texttt{\textcolor{uagreen}{\textbackslash usepackage}\{url\}} of (bij voorkeur) \texttt{\textcolor{uagreen}{\textbackslash usepackage}\{hyperref\}}!
\end{frame}

\subsection{\LaTeX}
\begin{frame}[fragile]
  \frametitle{En nu?}

  Nu vertellen we~\LaTeX\ dat we met een bibliografie werken:
  \begin{minted}{tex}
\bibliographystyle{plain}
\bibliography{file-1,file-2}
  \end{minted}
  \begin{alertblock}{Spaties}
  Geen spatie tussen opeenvolgende bestanden gebruiken. Extensies zijn ook niet nodig.
  \end{alertblock}
\end{frame}

\begin{frame}[fragile]
  \frametitle{En we citeren}

  \begin{minted}{tex}
  Dit is een citaat, zie~\cite{wiki:bibtex}.
  \end{minted}
\end{frame}

\begin{frame}[fragile]
  \frametitle{Werkwijze}

  Herinner je het two-pass systeem voor verwijzingen? Nu hebben we een four-pass:
  \begin{verbatim}
pdflatex document.tex
bibtex document.aux
pdflatex document.tex
pdflatex document.tex\end{verbatim}
\end{frame}

\begin{frame}[fragile]
  \frametitle{Na de eerste stap}
  \begin{verbatim}
LaTeX Warning: Citation `lamport94' on page 1 undefined
  on input line 21.
...
LaTeX Warning: There were undefined references.\end{verbatim}
\end{frame}

\begin{frame}[fragile]
  \frametitle{Na de tweede stap}
  \begin{verbatim}
This is BibTeX, Version 0.99c (Web2C 7.3.1)
The top-level auxiliary file: latex_source_code.aux
The style file: plain.bst
Database file #1: sample.bib\end{verbatim}
\end{frame}

\begin{frame}[fragile]
  \frametitle{Na de derde stap}
  \begin{verbatim}
LaTeX Warning: Label(s) may have changed.
Rerun to get cross-references right. 
  \end{verbatim}
\end{frame}

\begin{frame}
  \frametitle{In \TeX maker}

  Tijd voor experiment.
\end{frame}

\begin{frame}[fragile]
  \frametitle{Citeren in je eigen taal (oud)}

  \begin{minted}{tex}
\usepackage[fixlanguage]{babelbib}
\selectbiblanguage{dutch}

\bibliographystyle{babplain}
\bibliography{sample}
  \end{minted}
  Dus gebruik de juiste style. Maar er zitten nog steeds bugs in deze package!
\end{frame}

\begin{frame}[fragile]
  \frametitle{Citeren in je eigen taal (nieuw)}

  Om een tipje van de sluier van Bib\LaTeX\ op te lichten:

  \begin{minted}{tex}
\usepackage{biblatex}

\addbibresource{biblatex-examples.bib}
  \end{minted}
\end{frame}

\begin{frame}[fragile]
  \frametitle{Bib\LaTeX?}

  Bib\LaTeX\ is de toekomst, maar op dit moment is het nog niet ingeburgerd genoeg om er diep op in te gaan. Het is een \emph{package} die de functionaliteit van de zogeheten \emph{back-end} \texttt{bibtex} uitbreidt, maar tegelijkertijd is er een veel betere back-end ontwikkeld.
\end{frame}


\begin{frame}[fragile]
  \frametitle{Bib\LaTeX: samenvatting}

  Samenvatting:
  \begin{enumerate}
    \item gebruikt \verb|biber| of \verb|bibtex| als zogeheten back-end (\verb|biber| is hip en nieuw)
    \item veel betere ondersteuning van talen (o.a.\ door mezelf) 
    \item (t\'e) veel mogelijkheden
    \item beter ge\"implementeerd
    \item slechts three-pass
  \end{enumerate}
\end{frame}

\begin{frame}[fragile]
  \frametitle{Uitgewerkt voorbeeld (\texttt{.bib})}

  \inputminted{tex}{bib-example.bib}
\end{frame}

\begin{frame}[fragile]
  \frametitle{Uitgewerkt voorbeeld (\texttt{.tex})}

  \scriptsize
  \inputminted{tex}{bib-example.tex}
\end{frame}

\begin{frame}[fragile]
  \frametitle{Uitgewerkt voorbeeld met Bib\LaTeX\ (\texttt{.tex})}

  \scriptsize
  \inputminted{tex}{bib-example-biblatex.tex}
\end{frame}

%\begin{frame}[fragile]
%  \frametitle{Mijn excuses}
%
%  De package \package{babelbib} bevat nog bugs, ontdekte ik bij het maken van deze slides. Ik zal de maker contacteren om deze op te lossen. Tot dan moet je bijvoorbeeld je maanden meteen correct schrijven en editienummers voluit schrijven.
%\end{frame}
%
%
%\begin{frame}[fragile]
%  \frametitle{Hack}
%
%  \inputminted[firstline=9,lastline=19]{tex}{bib-example-extended.tex}
%\end{frame}
