\begin{frame}
  \frametitle{Waarom code typesetten?}

  Ik denk dat zowat elke wetenschapsstudent wel eens met code in aanraking komt: Matlab, R, \LaTeX, \cpp, \ldots

  Net zoals we \LaTeX\ schrijven in een editor met mooie kleurtjes en dergelijke kan het handig zijn om dit in je document ook te hebben:
  \begin{itemize}
    \item regelnummering;
    \item syntax highlighting;
    \item een fixed-width lettertype;
    \item automatisch code uit files halen;
    \item \ldots
  \end{itemize}
\end{frame}

\begin{frame}[fragile]
  \frametitle{Voorbeeld}

\begin{minted}[fontsize=\small,linenos=true,mathescape=true,frame=lines]{cpp}
string title = "Dit is een string";
/*
Gedefinieerd als $\pi=\lim_{n\rightarrow\infty}\frac{P_n}{d}$ met $P_n$ de omtrek
van een $n$-zijdige regelmatige veelhoek
ingeschreven in een cirkel met straal $d$.
*/
const double pi = 3.1415926235
\end{minted}
\end{frame}

\begin{frame}
  \frametitle{De mogelijkheden}

  In volgorde van oplopende awesomeness:
  \begin{enumerate}
    \item \texttt{verbatim} gebruiken;
    \item \package{listings};
    \item \package{minted}.
  \end{enumerate}
\end{frame}

\begin{frame}
  \frametitle{Alternatieven en gerelateerd}

  \begin{itemize}
    \item \package{algorithmic} voor \emph{pseudocode};
    \item \package{verbatim} als uitbreiding op~\texttt{\textcolor{uagreen}{\textbackslash verb}|$\langle$\textsl{tekst}$\rangle$|};
    \item \package{fancyvrb} floats voor code-omgevingen, wordt vaak intern gebruikt;
    \item \package{showexpl} \LaTeX-code met voorbeeld.
  \end{itemize}
\end{frame}

\begin{frame}[fragile]
  \frametitle{\texttt{verbatim}}

  Dit kennen we op zich al: \texttt{\textcolor{uagreen}{\textbackslash verb}|$\langle$\textsl{tekst}$\rangle$|}.

  Maar er is ook (zonder \package{verbatim}!):
  \begin{LTXexample}
\begin{verbatim}
dit is code
nieuwregels tellen
    spaties ook
\end{verbatim}
  \end{LTXexample}

  \begin{LTXexample}
\begin{verbatim*}
    spaties zichtbaar
\end{verbatim*}
  \end{LTXexample}
\end{frame}

\begin{frame}
  \frametitle{Voordelen}

  \begin{itemize}
    \item meteen te gebruiken, geen packages nodig;
    \item goed voor korte code of inline code;
    \item gemakkelijk.
  \end{itemize}
\end{frame}

\begin{frame}
  \frametitle{Nadelen}

  \begin{itemize}
    \item geen van de leuke dingen die we konden doen is mogelijk:
      \begin{itemize}
        \item syntax highlighting;
        \item regelnummers;
        \item files inputten;
        \item \LaTeX\ in commentaar.
      \end{itemize}
    \item geen line wrap.
  \end{itemize}
\end{frame}

\begin{frame}[fragile]
  \frametitle{\package{listings}}

  De oplossing voor al onze problemen.
  \begin{minted}{tex}
\usepackage{listings}

\begin{lstlisting}
  % commentaar
  we zullen eens \emph{proberen}
\end{lstlisting}
\end{minted}
\end{frame}

\begin{frame}
  \frametitle{Enkele conclusies}

  \begin{itemize}
    \item het lettertype is nog niet wat we willen;
    \item geen kleurtjes;
    \item geen regelnummering;
  \end{itemize}

  We moeten nog opties instellen:
  \texttt{\textcolor{uagreen}{\textbackslash lstset}\{$\langle$\textsl{opties$=$waarden}$\rangle$\}}
\end{frame}

\begin{frame}[fragile]
  \frametitle{Opties}

  \begin{description}[\texttt{numberstyle}]
    \item[\texttt{language}] de taal van de code, zie volgende slide voor ondersteunde talen;
    \item[\texttt{basicstyle}] opmaak die voor alle code geldt, we willen altijd \verb|basicstyle=\ttfamily|, of bijvoorbeeld \verb|basicstyle=\ttfamily\small|;
    \item[\texttt{numbers}] regelnummering;
    \item[\texttt{numberstyle}] opmaak voor regelnummers, bijvoorbeeld \verb|numberstyle=\footnotesize|;
    \item[\texttt{stepnumber}] sprongen in de regelnummers, bijvoorbeeld \verb|\stepnumber=5|;
  \end{description}
\end{frame}

\begin{frame}
  \frametitle{Talen}

  \small
  \begin{tabular}{lll}
    ABAP & IDL & Plasm           \\
    ACSL & inform & POV         \\
    Ada & Java & Prolog         \\
    Algol & JVMIS & Promela     \\
    Ant & ksh & Python          \\
    Assembler2 & Lisp & R       \\
    Awk & Logo & Reduce         \\
    bash & make & Rexx          \\
    Basic2 & Mathematica1 & RSL \\
    C & Matlab & Ruby           \\
    C++ & Mercury & S           \\
    Caml & MetaPost & SAS       \\
    Clean & Miranda & Scilab    \\
  \end{tabular}
\end{frame}

\begin{frame}
  \frametitle{Talen}

  \small
  \begin{tabular}{lll}
    Cobol & Mizar & sh          \\
    Comal & ML & SHELXL         \\
    csh & Modula-2 & Simula     \\
    Delphi & MuPAD & SQL        \\
    Eiffel & NASTRAN & tcl      \\
    Elan & Oberon-2 & TeX       \\
    erlang & OCL & VBScript     \\
    Euphoria & Octave & Verilog \\
    Fortran & Oz & VHDL         \\
    GCL & Pascal & VRML         \\
    Gnuplot & Perl & XML        \\
    Haskell & PHP & XSLT        \\
    HTML & PL/I
  \end{tabular}
\end{frame}

\begin{frame}[fragile]
  \frametitle{Meer opties}

  \begin{description}
    \item[\texttt{numbers}] positie van de regelnummers;
    \item[\texttt{showspaces}] zoals \texttt{\textcolor{uagreen}{\textbackslash verb}*|$\langle$\textsl{code}$\rangle$|};
    \item[\texttt{showstringspaces}] bovenstaande, specifiek voor strings;
    \item[\texttt{showtabs}] bovenstaande, specifiek voor tabs;
    \item[\texttt{breaklines}] line breaks (dit zal je meestal willen, dus \verb|breaklines=true|).
  \end{description}
\end{frame}

\begin{frame}[fragile]
  \frametitle{Syntax highlighting}

  \package{listings} is lui en doet uit zichzelf niks, daarom
  \begin{minted}{tex}
\lstset{%
  basicstyle=\ttfamily\small,
  keywordstyle=\color{blue},
  identifierstyle=\color{red},
  commentstyle=\color{green},
  stringstyle=\color{blue}
}
\end{minted}
\end{frame}

\begin{frame}[fragile]
  \frametitle{Float}

  Net zoals we tabellen en afbeeldingen in floats plaatsen zal dit voor code vaak ook een goed idee zijn, we gebruiken hiervoor de package \package{listing}:
  \small
  \begin{minted}{tex}
\usepackage{listing}

\listoflistings
\begin{listing}
  \lstinputlisting[language=cpp]{main.cpp}
  \caption{Een voorbeeldlisting}
  \label{listing:main-cpp}
\end{listing}
\end{minted}
\end{frame}

\begin{frame}
  \frametitle{Of rechtstreeks met \package{listings}}

  We bestuderen het hoofdstuk over captions in de manual. Dit is 4.9.

  \begin{alertblock}{Opgelet}
    Dit zijn geen echte floats! Maar we kunnen ze natuurlijk wrappen in floats, zonder captions te gebruiken.
  \end{alertblock}
\end{frame}

\begin{frame}
  \frametitle{\package{minted}}

  State of the art, maakt gebruik van een externe library (Pygments).

  Voordelen:
  \small
  \begin{itemize}
    \item enorm flexibele highlighting;
    \item veel betere lexers;
    \item Unicode-ondersteuning;
  \end{itemize}

  \normalsize Nadelen:
  \small
  \begin{itemize}
    \item geen line wrap;
    \item geen page breaks;
    \item niet bijster gemakkelijk (te installeren op Windows).
  \end{itemize}
\end{frame}

\begin{frame}
  \frametitle{\texttt{-{-}enable-write18}}

  \LaTeX\ mag niet zomaar bestanden schrijven enzo (het is namelijk mogelijk een virus te schrijven in \TeX), maar soms is het nu eenmaal nodig. We voegen hiervoor de optie \texttt{-{-}enable-write18} toe, zodat \LaTeX\ andere programma's mag oproepen en mag wegschrijven naar bestanden, wat nodig is om \package{minted} te gebruiken.

  \begin{alertblock}{Proof of concept}
    Zie \url{http://cseweb.ucsd.edu/~hovav/dist/texhack.pdf} voor de ge\"interesseerde. Maar don't panic, er bestaan geen echte.
  \end{alertblock}
\end{frame}

\begin{frame}[fragile]
  \frametitle{Voorbeeld van \package{minted}}

  \inputminted[fontsize=\scriptsize,linenos=true,mathescape=true,frame=lines]{tex}{minted-example.tex} 
\end{frame}

\begin{frame}[fragile]
  \frametitle{Oefening}

\lstset{
  basicstyle=\ttfamily,
  keywordstyle=\color{blue},
  identifierstyle=\color{red},
  commentstyle=\color{green},
  stringstyle=\color{blue},
  caption=\ttfamily\lstname
}

\lstinputlisting[language=c++]{main.cpp}
\end{frame}

\begin{frame}[fragile]
  \frametitle{Zoals hoofdstuk 4.9}

\lstset{
  basicstyle=\ttfamily,
  keywordstyle=\color{blue},
  identifierstyle=\color{red},
  commentstyle=\color{green},
  stringstyle=\color{blue},
}
    
  \begin{listing}[H]
    \lstinputlisting[language=c++,caption=]{main.cpp}
    \caption{Nu de caption in de float zelf}
  \end{listing}
\end{frame}
