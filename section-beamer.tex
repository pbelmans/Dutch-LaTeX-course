\begin{frame}
  \frametitle{Slides in \LaTeX?}

  Ja! U kijkt er overigens al 4 weken lang naar.

  Een gew\'eldige package: \package{beamer}. Voor iemand die beseft hoe \LaTeX\ echt werkt is \package{beamer} een prachtvoorbeeld van hoe krachtig \LaTeX\ is. Voor iemand die gewoon mooie presentaties wil maken daarentegen is het ook geweldig.
\end{frame}

\begin{frame}[fragile]
  \frametitle{Concept}

  Een aparte documentclass:

  \begin{minted}{tex}
 \documentclass{beamer}
  \end{minted}

  en frames per slide (in tegenstelling tot automatische pagina's):
  \begin{minted}{tex}
 \begin{frame}
   \frametitle{Titel}
 \end{frame}
  \end{minted}
\end{frame}

\begin{frame}
  \frametitle{Alles wat we kennen\ldots}

  \small
  werkt gewoon!
  \begin{gather}
    i\hbar\frac{\partial}{\partial t}\Psi=\hat{H}\Psi \\
    \left( \beta mc^2+\sum_{k=1}^3\alpha_k p_k c \right)\psi(\mathbf{x},t)=i\hbar\frac{\partial\psi(\mathbf{x},t)}{\partial t}
  \end{gather}
  \begin{itemize}
    \item opsommingen
    \item werken ook
  \end{itemize}
  Pas op met floats en verbatim.
\end{frame}

\begin{frame}[fragile]
  \frametitle{Titelslide}

  Zoals in \verb|article| werkt \texttt{\textcolor{uagreen}{\textbackslash author}\{\}} en \texttt{\textcolor{uagreen}{\textbackslash title}\{\}}. Om de titelslide z\'elf aan te maken:
  \begin{minted}{tex}
 \begin{frame}
   \titlepage
 \end{frame}
\end{minted}
\end{frame}

\begin{frame}[fragile]
  \frametitle{Structuur}

  De commando's \`a la \texttt{\textcolor{uagreen}{\textbackslash section}\{\}} werken nog steeds, maar geven geen rechtstreekse output! Maar \package{beamer} voorziet de mogelijkheid om hier \emph{automatisch} dingen mee te doen.
  \begin{minted}{tex}
 \AtBeginSection[]
 {
   \begin{frame}
 	   \frametitle{Overzicht}
 	   \tableofcontents[currentsection]
   \end{frame}
 }
  \end{minted}
\end{frame}

\begin{frame}[fragile]
  \frametitle{Handige omgevingen}

  \footnotesize
  Een hoop omgevingen die logisch uit \package{amsthm} volgen zijn voorgedefinieerd en hebben (mits de juiste stijl) een leuk resultaat:
  \begin{definition}
    Zoals hieruit blijkt. Merk op dat we hier geen controle hebben over de taal en we het dus alsnog zelf moeten doen. De ironie.
  \end{definition}

  \begin{alertblock}{Voorbeeld}
    Of hieruit.
  \end{alertblock}
  \begin{minted}{tex}
\begin{alertblock}{Voorbeeld}
  Of hieruit.
\end{alertblock}
  \end{minted}
\end{frame}

\begin{frame}[fragile]
  \frametitle{Overlays}

  \small
  \begin{enumerate}
    \item<1-> ook wel bekend als
    \item<2-> het automatisch uncoveren
    \item<3-> van slides
    \item<1-> maar ik doe dit niet vaak
  \end{enumerate}
  \begin{uncoverenv}<3->
  \begin{minted}{tex}
\begin{enumerate}
  \item<1-> ook wel bekend als
  \item<2-> het automatisch uncoveren
  \item<3-> van slides
  \item<1-> maar ik doe dit niet vaak
\end{enumerate}
\end{minted}
  \end{uncoverenv}
\end{frame}

\begin{frame}[fragile]
  \frametitle{Gebruik van overlays}

  Met \verb|\pause| werk je op de meest intu\"itieve manier, \`a la PowerPoint.

  Je kan dingen slechts 1 keer tonen met \texttt{<$\langle$\textsl{n}$\rangle$>}, tot een bepaalde slide \texttt{<-$\langle$\textsl{n}$\rangle$>}, of vanaf \texttt{<$\langle$\textsl{n}$\rangle$->}.

  Zie de manual voor het werken met omgevingen, maar normaal gezien is \verb|\pause| genoeg voor al je noden.

  \begin{alertblock}{Goede raad}
    Gebruik dit niet te veel! Heb je het gemist tijdens mijn lessen?
  \end{alertblock}
\end{frame}

\begin{frame}[fragile]
  \frametitle{Verbatim}

  Het waarom brengt ons te ver, maar gebruik \verb|[fragile]|:
  \begin{minted}{tex}
 \begin{frame}[fragile]
   \frametitle{Titel}

   \verb|verbatim|
 \end{frame}
  \end{minted}

  Anders ga je heel enge errors krijgen.
\end{frame}

\begin{frame}[fragile]
  \frametitle{Floats}

  Zul je minder vaak gebruiken, maar indien het toch gebeurt: met de \verb|[H]| placement modifier zou je jezelf problemen kunnen besparen.
\end{frame}

\begin{frame}
  \frametitle{Thema's}

  Met \texttt{\textcolor{uagreen}{\textbackslash usetheme}\{$\langle$\textsl{naam}$\rangle$\}} kan je andere thema's gebruiken. We bekijken klassikaal hoofdstuk 15 uit de manual.
\end{frame}

\begin{frame}
  \frametitle{\texttt{UniversiteitAntwerpen}}

  Geschreven door iemand van het departement Wiskunde--Informatica, zie ook \href{http://win.ua.ac.be/~nschloe/content/ua-beamer-theme}{\texttt{http://win.ua.ac.be/\textasciitilde nschloe/content/ua-beamer-theme}}.

  We zullen deze eens proberen te installeren.
\end{frame}
