\documentclass[a4paper]{article}
\usepackage{amsmath}
\usepackage[dutch]{babel}
\usepackage{hyperref}

\frenchspacing

\title{Opgave eerste les}
\author{$\langle$jouw naam$\rangle$}
\date{1 april 2011}

\newcommand\dash{\nobreakdash-\hspace{0pt}}

\begin{document}
\maketitle
\begin{abstract}
  We gaan nu zien of we iets hebben opgestoken de afgelopen ``les''. Bij wijze van oefening mag je dit document proberen na te maken. Enkele handige tips:
  \begin{enumerate}
    \item werk met voldoende tussenstappen\footnote{Dus vaak opslaan en builden.};
	\item leer Google goed te gebruiken;
	\item gebruik de slides;
	\item stel vragen.
  \end{enumerate}

  Dit stuk tekst hoort \emph{ook} bij de opgave overigens. Het enige dat vrij te kiezen is is je naam.
\end{abstract}

\tableofcontents
\clearpage

\section{Tekst}
Om te beginnen, een kort verhaal\ldots

\begin{quotation}
  Er was \'e\'ens in een land ver van hier een~$n$\dash dimensionale\footnote{Niet-wiskundigen negeren dit best om hun geestelijke gezondheid te beschermen.} robot met de welluidende naam se\~nor Guido Ev\'ereste. Zijn offici\"ele titel was echter prof.~dr.~ing.~Willy Willy.
  
  Hij at graag boeken, maar enkel de pagina's 33--37. Hij kookte deze op zijn gasvuur, water kookt zoals je weet op~$100\,^\circ\mathrm{C}$.
\end{quotation}
Een lijstje:
\begin{enumerate}
  \item dit is item 1;
  \item dit is item 2:
    \begin{itemize}
  	\item met een ongenummerd subitem;
  	\item of zelfs~\emph{twee}.
  \end{itemize}
\end{enumerate}

Volgende lijst is opgesteld op basis van~\href{http://nl.wikipedia.org}{nl.wikipedia.org}:
\begin{description}
  \item[aardappel] (\emph{Solanum tuberosum}) is een plant die ondergronds een energievoorraad in de vorm van zetmeel aanlegt. Het zetmeel wordt bewaard in de vorm van knollen, die eveneens aardappelen of aardappels worden genoemd. 
  \item[tractor] (ook~\emph{trekker}) is een voertuig dat speciaal is ontwikkeld voor gebruik in de landbouw, maar nu overal wordt gebruikt. 
\end{description}

\subsection{Titel}
Dit is een paragraaf.\\
Deze zin hoort bij dezelfde paragraaf maar begint toch op een nieuwe regel?!

\begin{flushright}
  Deze regel heeft een politieke voorkeur.
\end{flushright}

\begin{center}
  Deze iets minder.
\end{center}

\subsection{Andere titel}
Een moeilijke oefening:
\begin{quotation}
  \emph{Dit heeft nadruk \emph{binnen nadruk} dus opgelet}; wat een \textsl{rotopgave} met \textit{gemengde} \textsc{Opmaak}. En het is nog \texttt{niet gedaan} --- wat had je nu gedacht --- met de martelgang. Zo typ je \LaTeX\ op de enige correcte manier: \verb|\LaTeX|.
\end{quotation}

Nu zijn de zinnige dingen echt wel~\emph{voorbij}!

\section{Opdracht}
Zoek op wat \verb|\parindent|, \verb|\parskip| en \verb|\noindent| doen. Probeer het eens uit op dit document en bekijk het resultaat.

\end{document}
